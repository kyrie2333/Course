\documentclass[UTF8]{ctexart}
\usepackage{CJK}
\usepackage{amsmath}
\usepackage{geometry}
\usepackage{graphicx}
\usepackage{float}
\usepackage{algpseudocode}
\usepackage{algorithm}
\usepackage{algorithmicx}
\usepackage{caption,subcaption}
\usepackage{color}
\usepackage{mathtools}
\CTEXsetup[format={\Large\bfseries}]{section}
\title{Algorithm Design and Analysis: Assignment 3}
\author{钟赟 202028013229148}
\usepackage{geometry}
\geometry{left=2.5cm,right=2.5cm,top=2.5cm,bottom=2.5cm}

\newcommand{\fix}{\marginpar{FIX}}
\newcommand{\new}{\marginpar{NEW}}

\begin{document}
\maketitle
        
\section*{1 Monkeys and Bananas}

\section*{Solution}
\subsection*{1.1 Algorithm Description} 
首先把猴子和香蕉的位置按照从左到右的顺序编号,也即坐标从小到大。令每个猴子选择其顺序对应位置的香蕉,比如从左数第一只猴子拿从左数第一根香蕉、从左数第二只猴子拿从左数第二根香蕉等。

伪代码见 Algorithm 1。
\begin{algorithm}[h]
	\caption{MONKEYS\_AND\_BANANAS algorithm}
	\begin{algorithmic}[1]
		\Function{MONKEYS\_AND\_BANANAS}{$monkeys, bananas$}
		\State Sort $monkeys$ in ascending order; 
		\State Sort $bananas$ in ascending order; 
        \For{$i = 0$ to $bananas.length$}
            \State $time = $ max ($time, \lvert bananas[i] - monkeys[i] \rvert$);
        \EndFor
		\State \Return{$time$} ;
		\EndFunction
	\end{algorithmic}
\end{algorithm}

\subsection*{1.2 Greedy-choice Property } 
Greedy 选择策略为:每个猴子选择其对应位置的香蕉,例如:左起第一个猴子应该拿左起的第一根香蕉。

最优子结构:第 $i$ 只猴子拿第 $i$ 根香蕉。
\subsection*{1.3 Correctness Proof } 
假设根据 1.2 中的 Greedy 选择策略,所得到的时间不是最小的。

根据假设,存在 $i$,第 $i$ 个猴子没有拿到第 $i$ 根香蕉,设其拿了第 $j$ 根香蕉。则第 $j$ 只猴子拿了第 $k$ 根香蕉\ldots,以此类推。
这些乱序对构成了多个环。不失一般性,我们假设前 $i$ 个猴子和前 $i$ 根香蕉乱序,构成长度为 $i$ 的环,且这个环不能拆分成更小的环。我们只需证明,对长度为 $i$ 的环,其花费的时间大于按照顺序拿即可。

第 $1$ 个猴子不拿第1根香蕉,则必定存在另一根香蕉比第1根香蕉离它更近;此时对于其他所有的猴子来说,都不愿意跑更远的路去拿第一根香蕉。经过多步 Greedy 决策后,第 $i$ 只猴子只能去拿第一根香蕉,因为长度为 $i$ 的环,且这个环不能拆分成更小的环。我们只需证明,对长度为
第 $i$ 只猴子拿第1根香蕉的时间大于它们按顺序拿香蕉所需要的时间,最终的结果取最大值,也将大于它们按顺序拿香蕉的时间。

综上,乱序对中的猴子拿香蕉所需要的时间大于它们按顺序拿香蕉的时间,假设不成立,算法正确性得证。
\subsection*{1.4 Complexity}
排序的复杂度为 $\mathcal{O}(nlogn)$,数组遍历的复杂度为$\mathcal{O}(n)$,故时间复杂度为$\mathcal{O}(nlogn)$。

\section*{2 Job Schedule}


\section*{Solution}
\subsection*{2.1 Algorithm Description} 
算法描述:由于 PCs 数量无限制,则作业调度在 PCs 上耗时排序与在 supercomputer 上的耗时排序无关。因为不论按照什么顺序,supercomputer 总是在忙碌中,没有空闲,不影响总耗时的计算。因此我们考虑在 PCs 上的调度:每次决策时选择剩下 job 中,$f_i$ 最大的 job 来运行。
遍历每个 job,每个 job 运行的最少的时间为:之前所有 job 在 PCs 上累积剩余的时间、上一个 job 在 PCs 上剩余的时间、以及当前job的 $f_i$ ,三者取最大值。

伪代码见 Algorithm 2。
\begin{algorithm}[h]
	\caption{JOB\_SCHEDULE algorithm}
	\begin{algorithmic}[1]
		\Function{JOB\_SCHEDULE}{$p, f$}
		\State Sort $p$ and $f$ in descending order of $f$. 
        \For{$i = 0$ to $p.length$}
            \State $minTime += p[i]$;
            \If{$i == 0$}
                \State $remainTime = f[0]$;
            \Else
                \State $remainTime = $ max ($remainTime-p[i],$max($f[i-1]-p[i], f[i]$));
            \EndIf
        \EndFor
		\State \Return{$minTime+remainTime$} ;
		\EndFunction
	\end{algorithmic}
\end{algorithm}

\subsection*{2.2 Greedy-choice Property} 
根据 Greedy 思想,我们希望 supercomputer 和 PCs 尽量没有空闲。因此,我们选择先运行 $f_i$ 大的 job ,这样在 supercomputer 运行的时候,尽量让 PCs 不空闲。

最优子结构:每次决策选择剩下 job 中,$f_i$ 最大的 job。
\subsection*{2.3 Correctness Proof } 
假设存在一个最佳 job schedule 序列 $J_0, J_1,\dots, J_n$ ,其中对于 job $J_i$ 和 $J_{i+1}$,$f_i < f_{i+1}$。

设 pte 为在运行到某个 job 之前,在 supercomputer 上运行的总时间, fte 为当前运行的总时间。当对 job  $J_i$ 和 $J_{i+1}$ 做决策时,有 

$pte += p[i]; $

$fte_1 = max(fte, pte+f[i]); $

$pte += p[i+1]; $

$fte_2 = max(fte_1, pte+f[i+1]);$

如果我们交换job $J_i$ 和 $J_{i+1}$的运行顺序,则有

$pte += p[i+1]; $

$fte = max(fte_1, pte+f[i+1]); $

$pte += p[i]; $

$fte_2 = max(fte_1, pte+f[i]);$

改变运行顺序对 pte 没有影响。由于 $f[i] < f[i+1]$ ,那么在第一种运行顺序中有 $fte_2 > fte_1$,在第二种运行顺序中有 $fte_2 \ge fte_1$。因此,交换$J_i$ 和 $J_{i+1}$的运行顺序不会增大fte。

综上,贪心算法是正确的。
\subsection*{2.4 Complexity}
排序的复杂度为$\mathcal{O}(nlogn)$,遍历数组的复杂度为$\mathcal{O}(n)$,因此总复杂度为$\mathcal{O}(nlogn)$。

\section*{3 Cross the River}

\section*{Solution}
\subsection*{3.1 Algorithm Description} 
首先将所有人的体重从小到大排序。优先运送剩下的人中体重最重的人,如果他能够跟体重最轻的人一起乘船,则一起乘船,如果不能,他就单独乘船。

伪代码见 Algorithm 3 。
\begin{algorithm}[h]
	\caption{CROSS\_THE\_RIVER algorithm}
	\begin{algorithmic}[1]
		\Function{CROSS\_THE\_RIVER}{$num, limit, weights$}
        \State Sort $weights$ in ascending order. 
        \State $numBoat = 0$;
        \For{$(i = 0,j = num-1; i\le j; j--)$ }
            \State $sum = weights[i] + weight[j]$;
            \If{$sum < limit$}
                \State $i++$;
            \EndIf
        \State $numBoat ++$;
        \EndFor
		\State \Return{$numBoat$} ;
		\EndFunction
	\end{algorithmic}
\end{algorithm}
\subsection*{3.2 Greedy-choice Property} 
根据 Greedy 的思想,我们想要船的个数最少,最少的情况是乘船人数的 1/2 ,也就是说尽量让两个人乘坐一条船。为了最大化利用船的载量,我们应该优先让体重最大的乘客上船,并且尽量让其跟别人一起乘船,因此
考虑体重最轻的乘客,如果他不能跟体重最大的乘客一起上船,那么就没有乘客能够跟体重最大的乘客上船,他只能自己上船。因此可以保证尽量让两人一起乘船。

最优子结构:每次决策选择剩下的乘客中体重最大的乘客优先上船,在考虑体重最轻的乘客能否跟他一起上船。
\subsection*{3.3 Correctness Proof} 
假设根据 3.1 中的 Greedy 算法,得到的船的数量不是最少的。

根据假设知,存在两位乘客$i$和$j$,$i < j$,他们的体重之和 $weights[i] + weight[j] \le limit$,却分别单独乘坐了两条船。由于数组有序,得知 $weights[i] \le weights[j]$。
当对乘客 $j$ 进行决策时,$weights[j] < limits$ 且 $ weights[i] + weight[j] \le limit $,那么如果乘客$i$前的乘客都已经坐船走了,乘客 $j$ 会选择跟乘客 $i$ 一起乘船走;如果乘客 $i$ 前还有别的乘客,那么乘客 $j$ 会选择跟
当前体重最轻的乘客一起乘船走。因此,乘客 $j$ 不可能单独乘船走,与假设矛盾。

综上,假设不成立,该贪心算法是正确的。
\subsection*{3.4 Complexity}
排序的复杂度为$\mathcal{O}(nlogn)$,遍历数组的复杂度为$\mathcal{O}(n)$,因此总复杂度为$\mathcal{O}(nlogn)$。

\end{document}
